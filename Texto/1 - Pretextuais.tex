% Página de Dedicatória
\cleardoublepage
\thispagestyle{empty}

\begin{center}

\textbf{DEDICATÓRIA}
    
\end{center}



\vspace*{\fill}

\hspace*{7cm}
\begin{minipage}{\dimexpr\textwidth-8.2cm\relax}
Dedico este trabalho à minha família, pelo apoio incondicional, incentivo constante
e compreensão durante toda a minha trajetória acadêmica.
\end{minipage}



% Página de Agradecimentos
\cleardoublepage
\thispagestyle{empty}

\begin{center}

\textbf{AGRADECIMENTOS}
    
\end{center}

A Deus, por conceder saúde, força e sabedoria ao longo de toda esta caminhada, permitindo superar os desafios e alcançar mais esta etapa da minha formação acadêmica.

Aos meus queridos estagiários \textbf{Ayslan, Gustavo e Gabriel}, pela dedicação, comprometimento e colaboração ao longo do desenvolvimento deste trabalho, bem como pela troca de conhecimentos, apoio nas atividades realizadas e contribuição significativa para o andamento das etapas da pesquisa.

À minha família, pelo apoio incondicional, incentivo constante, compreensão e confiança depositada em mim durante todo o período de desenvolvimento deste trabalho. Sem esse suporte, esta conquista não seria possível.

Ao meu orientador, Prof. Dr. Douglas Bressan Riffel, pela orientação segura, paciência, disponibilidade e valiosas contribuições técnicas e acadêmicas, que foram fundamentais para a realização deste trabalho.

Aos professores do Programa de Pós-Graduação em Engenharia Elétrica da Universidade Federal de Sergipe, pelos ensinamentos transmitidos, pela dedicação e pela contribuição para a minha formação acadêmica e profissional.

Aos colegas de curso, pelo companheirismo, troca de conhecimentos e apoio mútuo ao longo desta trajetória.

À Universidade Federal de Sergipe e ao Programa de Pós-Graduação em Engenharia Elétrica, pela infraestrutura e pelas condições oferecidas para o desenvolvimento desta pesquisa.

Por fim, a todos que, direta ou indiretamente, contribuíram para a realização deste trabalho, meus sinceros agradecimentos.






% Página de Epígrafe
\cleardoublepage
\thispagestyle{empty}

\begin{center}
\textbf{EPÍGRAFE}
\end{center}

\vspace*{\fill}

\noindent
\hspace*{8cm}
\begin{minipage}{\dimexpr\textwidth-8cm\relax}
\textit{“A verdadeira viagem de descobrimento não consiste em procurar novas paisagens, mas em ter novos olhos.”}

\begin{flushright}
- Marcel Proust
\end{flushright}
\end{minipage}





% Página de Resumo
\cleardoublepage
\thispagestyle{empty}

\begin{center}
\textbf{RESUMO}
\end{center}


O aumento da demanda global por energia, aliado à escassez de combustíveis fósseis e às preocupações ambientais relacionadas às emissões de gases de efeito estufa, tem impulsionado a busca por fontes alternativas de energia, com destaque para as fontes renováveis. Nesse contexto, a energia solar fotovoltaica se consolida como uma solução estratégica, devido à sua natureza limpa, ampla disponibilidade e à expressiva redução dos custos dos sistemas ao longo dos últimos anos. No Brasil, embora a matriz elétrica seja majoritariamente renovável, a geração solar ainda apresenta significativo potencial de expansão, favorecida pelos elevados índices de irradiação solar e pelos incentivos governamentais.

No Estado de Sergipe, a energia solar tem se destacado como uma alternativa promissora, impulsionada pelas condições climáticas favoráveis e por políticas de incentivo ao setor. O Sergipe Parque Tecnológico (SergipeTec), dispondo de uma área de aproximadamente 3.000 m² destinada a estacionamento e com elevada incidência solar, implantou uma usina fotovoltaica do tipo carport solar, que alia a geração de energia elétrica à otimização do uso do espaço urbano.

Diante da carência de estudos específicos voltados à avaliação do desempenho e da viabilidade econômica de usinas fotovoltaicas do tipo carport, esta dissertação propõe a análise da eficiência energética, do desempenho operacional e da viabilidade econômica da usina carport solar instalada no SergipeTec. A avaliação é realizada por meio da análise de dados reais de geração, utilizando indicadores-chave de desempenho (Key Performance Indicators – KPIs), calculados a partir do monitoramento dos 53 microinversores do sistema. Os resultados obtidos visam validar o funcionamento da usina, mensurar seus benefícios energéticos e econômicos, bem como avaliar a confiabilidade do sistema, contribuindo para a tomada de decisões e para a disseminação de soluções sustentáveis em ambientes tecnológicos e inovadores.

\noindent\textbf{Palavras Chave:} Energia solar fotovoltaica; Carport solar; Eficiência energética; Indicadores-chave de desempenho; Microinversores.





% Página de Abstract
\cleardoublepage
\thispagestyle{empty}

\begin{center}
\textbf{ABSTRACT}
\end{center}

The growing global demand for energy, combined with the scarcity of fossil fuels and environmental concerns related to greenhouse gas emissions, has driven the search for alternative energy sources, particularly renewable ones. In this context, photovoltaic solar energy has emerged as a strategic solution due to its clean nature, wide availability, and the significant reduction in system costs over recent years. In Brazil, although the electric power matrix is predominantly renewable, solar generation still presents considerable potential for expansion, supported by high solar irradiation levels and government incentive policies.

In the State of Sergipe, solar energy has gained prominence as a promising alternative, driven by favorable climatic conditions and sector incentive policies. The Sergipe Technology Park (SergipeTec), which has an area of approximately 3,000 m² allocated for parking and high solar incidence, implemented a photovoltaic power plant based on the solar carport model, combining electricity generation with the optimization of urban space use.

Given the lack of specific studies focused on evaluating the performance and economic feasibility of solar carport photovoltaic systems, this dissertation proposes an analysis of the energy efficiency, operational performance, and economic viability of the solar carport plant installed at SergipeTec. The assessment is carried out through the analysis of real generation data using key performance indicators (KPIs), calculated from the monitoring of the system’s 53 microinverters. The results aim to validate the operation of the plant, quantify its energy and economic benefits, and assess system reliability, contributing to decision-making processes and to the dissemination of sustainable solutions in innovative and technological environments.

\noindent\textbf{Keywords:} Photovoltaic solar energy; Solar carport; Energy efficiency; Key performance indicators; Microinverters.

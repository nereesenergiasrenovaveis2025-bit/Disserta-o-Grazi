% Página de Dedicatória
\cleardoublepage
\thispagestyle{empty}

\begin{center}

\textbf{DEDICATÓRIA}
    
\end{center}



\vspace*{\fill}

\hspace*{7cm}
\begin{minipage}{\dimexpr\textwidth-8.2cm\relax}
Dedico este trabalho em memória do meu pai, José Fernandes Monteiro, meu herói, que sempre me incentivou a continuar os estudos,
 e dedico também em memória da minha avó, Júlia Maria de Almeida Oliveira, que sempre me incentivou a correr atrás dos meus sonhos. Tenho saudades de tudo que passei ao lado de vocês. Como eu queria agora poder abraçá-los e agradecer por tudo, pois, sem vocês, este trabalho e muitos dos meus sonhos não se realizariam. Te amo! 
\end{minipage}



% Página de Agradecimentos
\cleardoublepage
\thispagestyle{empty}

\begin{center}

\textbf{AGRADECIMENTOS}
    
\end{center}

 Agradeço a todos da minha família pela paciência e compreensão. Nesse sentido, destaco minha mãe, Nilma, pelo incentivo e estímulo para enfrentar as barreiras da vida; meu irmão, Rodrigo, pelo apoio de sempre; minha filha, Amanda Grazyelle, por ser carinhosa e dedicada, tenho orgulho de ser sua mãe, pois com você aprendi o que é o amor verdadeiro; e meu marido, Paulo Estevão, pelo companheirismo, carinho e amor.

Agradeço aos colegas de trabalho do Sergipe Parque Tecnológico (SergipeTec), especialmente à diretoria, a José Augusto, Anízio, Luciana, Carlos e ao coordenador Rennan, pelo incentivo à realização do meu mestrado, pelas condições oferecidas e por permitirem minha dedicação à linha de pesquisa intitulada como \textit{Análise de Desempenho e Eficiência Energética de Usinas Fotovoltaicas}, agradeço também aos estagiários Ayslan, Gabriel e Gustavo, pela dedicação, comprometimento e colaboração, bem como pela troca de conhecimentos ao longo do desenvolvimento desta pesquisa.

Agradeço à Universidade Federal de Sergipe e ao Programa de Pós-Graduação em Engenharia Elétrica, por terem me proporcionado essa experiência durante a minha caminhada. Ao meu orientador, Prof. Dr. Douglas Bressan Riffel, agradeço pelos conselhos, ensinamentos, disponibilidade e valiosas contribuições que foram fundamentais para a realização desta pesquisa; Agradeço também aos professores que desempenharam com dedicação as aulas ministradas e pela contribuição para a minha formação.

Agradeço a todos que contribuíram direta e indiretamente para a minha trajetória, até as críticas foram essenciais para me incentivar a não desistir dos meus sonhos. O caminho seguido foi longo repleto de dificuldades para chegar até aqui. 

Por fim, agradeço a Deus por me guiar e colocar pessoas tão especiais ao meu lado.



% Página de Epígrafe
\cleardoublepage
\thispagestyle{empty}

\begin{center}
\textbf{EPÍGRAFE}
\end{center}

\vspace*{\fill}

\noindent
\hspace*{8cm}
\begin{minipage}{\dimexpr\textwidth-8cm\relax}
\textit{“Dizem que a vida é para quem sabe viver,
mas ninguém nasce pronto. \\
A vida é para quem é corajoso o suficiente para se arriscar
e humilde o bastante para aprender.”}

\begin{flushright}
- Clarice Lispector
\end{flushright}
\end{minipage}





% Página de Resumo
\cleardoublepage
\thispagestyle{empty}

\begin{center}
\textbf{RESUMO}
\end{center}


O aumento da demanda global por energia, aliado à escassez de combustíveis fósseis e às preocupações ambientais relacionadas às emissões de gases de efeito estufa, tem impulsionado a busca por fontes alternativas de energia, com destaque para as fontes renováveis. Nesse contexto, a energia solar fotovoltaica se consolida como uma solução estratégica, devido à sua natureza limpa, ampla disponibilidade e à expressiva redução dos custos dos sistemas ao longo dos últimos anos. No Brasil, embora a matriz elétrica seja majoritariamente renovável, a geração solar ainda apresenta significativo potencial de expansão, favorecida pelos elevados índices de irradiação solar e pelos incentivos governamentais.

No Estado de Sergipe, a energia solar tem se destacado como uma alternativa promissora, impulsionada pelas condições climáticas favoráveis e por políticas de incentivo ao setor. O Sergipe Parque Tecnológico (SergipeTec), dispondo de uma área de aproximadamente 3.000 m² destinada a estacionamento e com elevada incidência solar, implantou uma usina fotovoltaica do tipo carport solar, que alia a geração de energia elétrica à otimização do uso do espaço urbano.

Diante da carência de estudos específicos voltados à avaliação do desempenho e da viabilidade econômica de usinas fotovoltaicas do tipo carport, esta dissertação propõe a análise da eficiência energética, do desempenho operacional e da viabilidade econômica da usina carport solar instalada no SergipeTec. A avaliação é realizada por meio da análise de dados reais de geração, utilizando indicadores-chave de desempenho (Key Performance Indicators – KPIs), calculados a partir do monitoramento dos 53 microinversores do sistema. Os resultados obtidos visam validar o funcionamento da usina, mensurar seus benefícios energéticos e econômicos, bem como avaliar a confiabilidade do sistema, contribuindo para a tomada de decisões e para a disseminação de soluções sustentáveis em ambientes tecnológicos e inovadores.

\noindent\textbf{Palavras Chave:} Energia solar fotovoltaica; Carport solar; Eficiência energética; Indicadores-chave de desempenho; Microinversores.





% Página de Abstract
\cleardoublepage
\thispagestyle{empty}

\begin{center}
\textbf{ABSTRACT}
\end{center}

The growing global demand for energy, combined with the scarcity of fossil fuels and environmental concerns related to greenhouse gas emissions, has driven the search for alternative energy sources, particularly renewable ones. In this context, photovoltaic solar energy has emerged as a strategic solution due to its clean nature, wide availability, and the significant reduction in system costs over recent years. In Brazil, although the electric power matrix is predominantly renewable, solar generation still presents considerable potential for expansion, supported by high solar irradiation levels and government incentive policies.

In the State of Sergipe, solar energy has gained prominence as a promising alternative, driven by favorable climatic conditions and sector incentive policies. The Sergipe Technology Park (SergipeTec), which has an area of approximately 3,000 m² allocated for parking and high solar incidence, implemented a photovoltaic power plant based on the solar carport model, combining electricity generation with the optimization of urban space use.

Given the lack of specific studies focused on evaluating the performance and economic feasibility of solar carport photovoltaic systems, this dissertation proposes an analysis of the energy efficiency, operational performance, and economic viability of the solar carport plant installed at SergipeTec. The assessment is carried out through the analysis of real generation data using key performance indicators (KPIs), calculated from the monitoring of the system’s 53 microinverters. The results aim to validate the operation of the plant, quantify its energy and economic benefits, and assess system reliability, contributing to decision-making processes and to the dissemination of sustainable solutions in innovative and technological environments.

\noindent\textbf{Keywords:} Photovoltaic solar energy; Solar carport; Energy efficiency; Key performance indicators; Microinverters.

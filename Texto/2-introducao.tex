\renewcommand{\chaptername}{} % remove "Capítulo"

\chapter{INTRODUÇÃO}
%\addcontentsline{toc}{chapter}{INTRODUÇÃO}


\label{cap:intro}


O crescente aumento da demanda global por energia, a escassez de combustíveis fósseis e as preocupações ambientais, como as emissões de gases de efeito estufa, têm impulsionado a busca por fontes alternativas de energia, destacando-se as energias renováveis (GULATI et al., 2023). Entre as energias renováveis, a energia solar fotovoltaica se consolidou como uma das principais soluções, devido à sua natureza limpa, ampla disponibilidade e a significativa redução dos custos dos módulos e sistemas ao longo da última década. 

No Brasil, apesar da matriz elétrica ser predominantemente renovável, a participação da energia solar ainda apresenta um vasto potencial de crescimento, impulsionado por altos níveis de irradiação solar e devido à redução dos preços dos módulos fotovoltaicos e aos incentivos governamentais (PARENTE, 2021). 


A energia solar tem ganhado cada vez mais espaço no Estado de Sergipe como uma fonte promissora e sustentável de geração de energia, apresentando um grande potencial para o aproveitamento da energia solar devido à sua localização geográfica privilegiada, com alta incidência solar ao longo do ano, somado às políticas e incentivos implementados,  as quais vêm impulsionando os investimentos no setor. Essas medidas incluem a criação de linhas de financiamento com taxas de juros favoráveis, programas de incentivo fiscal, simplificação dos processos burocráticos para a obtenção de licenças e autorizações, além da promoção de campanhas de conscientização sobre os benefícios da energia solar (SANTOS, et al., 2024). 
 

O Sergipe Parque Tecnológico (SergipeTec) possuía uma área livre para estacionamento de aproximadamente 3.000 m², sendo um espaço com uma irradiação solar diária média anual de 5,29 $\frac{kWh}{(m^2.  dia)}$  (CRESESB,2025). Essa área disponível foi então utilizada para a criação da usina solar fotovoltaica para alimentação de energia do parque. Dessa forma, surgiu a ideia da utilização do modelo carport solar que une a geração de energia fotovoltaica à otimização do uso do espaço. 


A problemática central reside na carência de estudos específicos que avaliem a eficiência energética e a viabilidade econômica de usinas solares do tipo carport solar, nesse contexto, esta proposta de dissertação analisa a viabilidade, o desempenho e eficiência energética da usina de estrutura carport Solar localizada no Sergipe Parque Tecnológico (SergipeTec). A motivação para a escolha do tema surgiu da oportunidade de realizar um estudo analisando os dados gerados pela usina instalada, justifica-se pela necessidade de integrar soluções sustentáveis em ambientes inovadores e tecnológicos. Com o objetivo de validar o funcionamento da usina, mensurar seus benefícios energéticos, economicos financeiros e verificar a confiabilidade do sistema atraves da analise de desempenho dos 53 microinversores. Para isso, são definidos \textit{Key Performance Indicators (KPIs)}, para o português indicadores chave de desempenho, que serão calculados e monitorados através da coleta, tratamento e análise de dados gerados, permitindo avaliar sua confiabilidade e apoiar a tomada de decisões.

\section{OBJETIVOS}
\label{objetivos}


\subsection{Objetivo Geral}


%Caso queira algo comentado, usar % o que estiver depois não aparecerá no arquivo

Esta proposta de dissertação almeja o desenvolvimento de um estudo de caso para análise de desempenho e eficiência energética da usina fotovoltaica em modelo carport solar, instalada no Sergipe Parque Tecnológico (SergipeTec). Portanto, será utilizada como base a metodologia dos indicadores chave de desempenho (KPIs) para verificar a confiabilidade do sistema e auxiliar na detecção de possíveis falhas.

\subsection{Objetivos Específicos}

Para atingir o objetivo geral desta proposta, são elencados os seguintes objetivos específicos:

\begin{itemize}
    
    \item Escolha de indicadores chave de desempenho (KPIs); 
    \item Coleta de dados e análise de dados gerados na usina solar;
    \item Avaliação de desempenho da usina do SergipeTec;
    \item Verificar a eficiência energética e a confiabilidade do sistema;
    \item Modelar o sistema fotovoltaico em ambiente de simulação computacional para obtenção dos possíveis impactos do sombreamento sobre os painéis solares;

    
\end{itemize}

\renewcommand{\chaptername}{} % remove "Capítulo"
\chapter{REVISÃO BIBLIOGRÁFICA}

A crescente participação dos sistemas fotovoltaicos (FV) na matriz energética mundial e brasileira evidencia a necessidade de monitoramento rigoroso para garantir sua confiabilidade e eficiência econômica (LINDIG et al., 2024; REDISKE et al., 2024). Tendo em vista a longa vida útil desses sistemas, que pode ultrapassar 20 anos, a análise contínua do seu desempenho é essencial para manter a geração de eletricidade conforme o esperado pelos investidores e para assegurar o retorno sobre o investimento (REDISKE et al., 2024; OLIVEIRA, 2021). Este processo de avaliação é fundamental, pois falhas podem ocorrer com periodicidade em períodos estimados de anos, por planta, impactando a performance do sistema (OLIVEIRA, 2021). A avaliação precisa requer uma metodologia robusta de análise que vá além da mera observação, utilizando indicadores padronizados que consigam quantificar a saúde e a performance dos ativos FV (REDISKE et al., 2024; LINDIG et al., 2024).

Para realizar a quantificação do desempenho são utilizados os KPIs, que são ferramentas cruciais para a avaliação da eficiência energética e do estado operacional dos sistemas FV, sendo frequentemente utilizados em acordos contratuais e análises técnicas (LINDIG et al., 2024). A International Electrotechnical Commission (IEC), por meio da norma IEC 61724 e suas partes, é a principal referência na definição de terminologia e métodos para monitoramento e análise de dados em geradores fotovoltaicos (LINDIG et al., 2024; REDISKE et al., 2024).

Dentre os indicadores mais importantes para a análise de desempenho energético e medição de perdas, destacam-se: Performance Ratio (PR) ou Taxa de Desempenho (TD) (GULATI et al., 2021); Fator de Capacidade (Capacity Factor - CF) (REDISKE et al., 2024; OLIVEIRA, 2021; GULATI et al., 2021); Produtividade do Sistema (Yf) ou Rendimento Final, que é útil para a comparação de sistemas de potências distintas (PARENTE, 2021; REDISKE et al., 2024); Rendimento de Referência (Yr), essencial para o cálculo do PR (REDISKE et al., 2024); Perdas do Sistema (Ls).

Para que a análise de desempenho e a estimativa de KPIs sejam confiáveis, é imprescindível a alta qualidade dos dados de entrada (LINDIG et al., 2024). O processo de cálculo de KPIs, como o Performance Ratio (PR) e o Fator de Utilização de Capacidade (CUF), requer um procedimento rigoroso de coleta e tratamento de dados, o que é denominado Rotinas de Qualidade de Dados (DQR) (GULATI et al., 2021; LINDIG et al., 2024). A confiabilidade dos KPIs depende da: 1° Precisão, ligada à qualidade das medições; 2º Completude, referente à ausência de dados faltantes; e 3° Consistência, que garante a integridade dos dados em diferentes fontes (LINDIG et al., 2024). A perda de dados e a variabilidade climática podem introduzir vieses significativos nos cálculos do PR, afetando os resultados contratuais e as decisões de Operação de Manutenção O\&M (LINDIG et al., 2024).

Outro aspecto importante para o estudo da eficiência de um gerador, é a gestão eficaz da Operação e Manutenção (O\&M) do mesmo, sendo vital para manter a performance esperada e assegurar a eficiência energética a longo prazo (REDISKE et al., 2024). Os KPIs de O\&M se concentram em medir a qualidade do serviço prestado, cobrindo tempo de resposta e disponibilidade de peças, sendo complementares aos KPIs de desempenho energético (REDISKE et al., 2024).

Para completar a análise de desempenho e realizar o estudo da viabilidade econômica, onde são utilizados indicadores como Levelized Cost of Energy (LCOE) (em portugês Custo Nivelado de Energia) e Período de Retorno (Payback Period), que são cruciais para determinar a viabilidade financeira, se faz necessário comparar a eficiência de diferentes tecnologias e configurações de projeto, o que auxilia na tomada de decisão.

Efetuar a comparação de diferentes tipos estruturas e diferentes modelos de Inversores é necessária, como pode ser demonstrado que a instalação de microinversores proporciona um desempenho superior em cerca de 8\% em relação ao inversor com string em alguns indicadores calculados (Yf, PR, FC) (PARENTE, 2021). Esse ganho é atribuído, principalmente, à individualização dos módulos, já com relação à estruturas, a tecnologia Carport (estacionamento solar) representa uma solução sustentável e estratégica para a infraestrutura de carregamento de veículos elétricos (VEs), pois utiliza áreas de estacionamento existentes para propiciar sombreamento, gerar energia limpa e integrar estações de recarga (RIAKHI; KHALDOUN, 2021; KULIK, 2024).


% para escrever um texto em italico \textit{TEXTO AQUI}
% para citar uma referencia \cite{lantada2012novel}
% \ref{CHAVE}  Aqui se referencia a secção que está o \label{CHAVE}, 



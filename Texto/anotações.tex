Resultados alterar o texto, algumas ideias


1 - Ao final, os resultados obtidos permitirão avaliar a viabilidade técnica, econômica e ambiental do uso de carports solares no SergipeTec, identificando as condições em que o sistema apresenta melhor desempenho. Essas conclusões possibilitarão o estabelecimento de diretrizes para aprimorar o modelo, subsidiar sua expansão em outras áreas do parque e fundamentar novas pesquisas e publicações científicas. Espera-se também que o trabalho de mestrado contribua para decisões mais assertivas e para a redução dos custos energéticos, fortalecendo a sustentabilidade do parque tecnológico.
2 - Os resultados serão comparados a partir de critérios de eficiência e sensibilidade, a fim de identificar os fatores que mais influenciam o desempenho energético.
3 - A análise da confiabilidade operacional é fundamental, uma vez que se busca propor um modelo de carport solar que seja eficiente, resistente e aplicável a diferentes áreas do SergipeTec.

Criar a parte de desenvolvimento do projeto 
- COLETA DE DADOS 
- CATALOGAÇÃO DA USINA SOLAR FOTOVOLTAICA 
- MODELAGEM DE SOMBREAMENTO PVSOL


com o objetivo de verificar a confiabilidade do modelo de carport solar e a economia financeira resultante e demonstrar a validação de seu funcionamento e coleta de dados para discussão dos resultados. 
Por fim, os resultados serão utilizados para comparar o desempenho dos 53 microinversores presentes na usina solar.



Esta proposta de dissertação estrutura-se como um estudo abrangente sobre o desempenho, a eficiência energética e a viabilidade econômica da usina de Carport Solar já instalada no Sergipe Parque Tecnológico (SergipeTec), sendo as atividades executadas até o momento, A1 (referencial teórico), A2 (mapeamento da usina solar fotovoltaica, com as identificações dos 53 microinversores e com as identificações dos 210 módulos fotovoltaicos, foram identificados também das portas de cada microinversor conectado em cada módulo fotovoltaico), além da A3 (monitoramento da geração da usina por seis meses, iniciando no mês de março até agosto de 2025, através da plataforma Solarman Smart),A4 (Fundamentação teórica sobre indicadores-chave de desempenho atraves de atigos publicados) e em andamento a A5 (levantamento dos dados dos indicadores chave de desempenho mais utilizados e a criação de uma
planilha Excel para a preparação para os cálculos de KPIs). Estas atividades forneceram uma base necessária para compreender o funcionamento do sistema e identificar parâmetros relevantes para as próximas fases da pesquisa.


Diante do que foi apresentado e por se tratar de uma pesquisa em andamento. As atividades subsequentes contemplam desde o aprofundamento teórico sobre indicadores-chave de desempenho (KPIs), simulações de sombreamento em ambiente 3D via PV*SOL e levantamentos complementares de dados das faturas de energia, até a consolidação dos cálculos de eficiência, análises financeiras e geração dos principais KPIs por microinversor. A integração dos resultados técnicos e econômicos permitirá uma avaliação da performance da usina e do seu retorno energético e financeiro ao longo dos anos.

A elaboração dos gráficos e relatórios finais, bem como a análise consolidada de eficiência, constituirá uma síntese abrangente dos achados do estudo. Por fim, a fase de escrita da dissertação e a posterior defesa integrarão todas as evidências, cálculos, simulações e interpretações construídas ao longo de quase dois anos de pesquisa.

Assim, esta proposta de dissertação demonstra o avanço já alcançado te o momento e mostra o planejamento detalhado das atividades que ainda serão desenvolvidas até 2027. 


 
GRADAR O CRONOGRAMA DE ATIVIDADES COM DATAS:
 - A1: Referencial teórico e revisão bibliográfica acerca de usinas solares carport, usina solares com microinversores e indicadores chave de desempenho (KPIs); 3 meses (março junho2025);
- A2: Mapeamento da usina de Carpor Solar; 4 meses(março julho2025);
- A3: Monitoramento da Usina e  coleta de dados de geração via Solarman Smart; 6 meses (Março Agosto2025);
- A4: Estudo da fundamentação teórica sobre indicadores-chave de desempenho (KPIs); 3 meses (Agosto Novembo2025);
- A5: Levantamento dos Dados dos indicadores-chave de desempenho (KPIs) em planilha Excel, preparação para cálculos de KPIs; 4 meses (Novembro Março2026);
- A6: Cálculos dos indicadores-chave de desempenho (KPIs), cálculo de payback e economia anual; 4 meses (Fevereiro Junho2026);
- A7: Levantamento das Faturas de Energia; coleta de infomações nas faturas de energia eletrica, para um períodos de 1 ano antes e 1 ano após a primeira etapa da usina. 2 meses (junho agosto2026);
- A8: Simulação de Sombreamento (PV*SOL), modelagem 3D da usina, análise de sombreamento e impacto no desempenho; 03 meses;(Agosto Novembro2026);
- A9: Levantamento bibliográfico sobre energia solar, sombreamento, microinversores, KPIs, análises financeiras, usinas FV; 4 meses (Agosto Dezembro2026);
- A10: Análise Financeira e Econômica; gráficos de desempenho, KPIs por microinversor, gráficos financeiros (payback); 2 meses (Agosto Outubro2026);
- A11: Geração de Gráficos de KPIs (Excel e Power BI) e gráficos de retrorno financeiro;03 meses (Setembro Dezembro2026);
- A12:Relatório de Eficiência; consolidação dos resultados técnicos e financeiros, análise final da usina;02 meses (Outubro Dezembro2026);
- A13: Escrita da dissertação; 12 meses (Janeiro2026 Janeiro2027);
- A14: Defesa de dissertação. 1 meses (Março2027);

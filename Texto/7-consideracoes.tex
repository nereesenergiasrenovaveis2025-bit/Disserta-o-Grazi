\renewcommand{\chaptername}{} % remove "Capítulo"
\chapter{Conclusões}
%\addcontentsline{toc}{chapter}{CONSIDERAÇÕES FINAIS}
\label{cap:consideracoes}


Esta proposta de dissertação estrutura-se como um estudo abrangente sobre o desempenho, a eficiência energética e a viabilidade econômica da usina de carport solar já instalada no Sergipe Parque Tecnológico (SergipeTec), sendo as atividades executadas até o momento, as supracitadas no cronograma de atividades. Estas atividades forneceram uma base necessária para compreender o funcionamento do sistema e identificar parâmetros relevantes para as próximas fases da pesquisa.


Diante do que foi apresentado e por se tratar de uma pesquisa em andamento. As atividades subsequentes contemplam desde o aprofundamento teórico sobre indicadores chave de desempenho (KPIs), simulações de sombreamento em ambiente 3D via PV*SOL e levantamentos complementares de dados das faturas de energia, até a consolidação dos cálculos de eficiência energética, análises financeiras e geração dos principais KPIs por microinversor. A integração dos resultados técnicos e econômicos permitirá uma avaliação da performance da usina e do seu retorno energético e financeiro ao longo dos anos.

Em síntese, a elaboração dos gráficos e relatórios finais, bem como a análise consolidada de eficiência, constituirá um avanço significativo no estudo. Por fim, a fase de escrita da dissertação e a posterior defesa integrarão todas as evidências, cálculos, simulações e interpretações construídas ao longo de quase dois anos de pesquisa.

Assim, esta proposta de dissertação demonstra o avanço já alcançado até o momento e mostra o planejamento detalhado das atividades que ainda serão desenvolvidas até 2027. 





